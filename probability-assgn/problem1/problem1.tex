\documentclass[journal,12pt,twocolumn]{IEEEtran}
\usepackage{graphicx}
\graphicspath{{./figs/}}{}
\usepackage{amsmath,amssymb,amsfonts,amsthm}
\newcommand{\myvec}[1]{\ensuremath{\begin{pmatrix}#1\end{pmatrix}}}
\usepackage{listings}
\usepackage{watermark}
\usepackage{titlesec}
\usepackage{amsmath}
\newcommand{\Mod}[1]{\ (\mathrm{mod}\ #1)}
\let\vec\mathbf

\titlespacing{\subsection}{0pt}{\parskip}{-3pt}
\titlespacing{\subsubsection}{0pt}{\parskip}{-\parskip}
\titlespacing{\paragraph}{0pt}{\parskip}{\parskip}
\newcommand{\figuremacro}[5]{
    
}
\lstset{
frame=single, 
breaklines=true,
columns=fullflexible
}
\thiswatermark{\centering \put(0,-105.0){\includegraphics[scale=0.08]{logo.jpg}} }

\sloppy
\title{\mytitle}
\title{
Probability Assignment
}
\author{T.Sai Raghavendra(FWC22087)}
\begin{document}
\maketitle
\tableofcontents
\bigskip


\section{\textbf{Problem1}}
\subsection{\textbf{Question}}
A die has two faces each with number ‘1’, three faces each with number ‘2’ and
one face with number ‘3’. If die is rolled once, determine\\
(i) P(2) \\(ii) P(1 or 3) \\(iii) P(not 3)\\
\subsection{\textbf{Solution}}
Total number of faces = 6

i) Number faces with number '2' = 3
\begin{center}
	P(2) = $\frac{3}{6}$ = $\frac{1}{2}$
\end{center}
	
ii) P(1 + 3) = P(1) + P(3) - P(13)
\begin{center}
	Number faces with number '1' = 2\\
	P(1) = $\frac{2}{6}$\\
	Number faces with number '3' = 1\\
	P(3) = $\frac{1}{6}$\\
	P(13) = 0\\
	So, P(1 + 3) = $\frac{2}{6}$ + $\frac{1}{6}$ - 0 = $\frac{3}{6}$ = $\frac{1}{2}$
\end{center}
	
iii) P(3') = 1 - P(3)
\begin{center}
	Number faces with number '3' = 1\\
	P(3) = $\frac{1}{6}$\\ 
	P(3') = 1 - $\frac{1}{6}$ = $\frac{5}{6}$
\end{center}
\end{document}