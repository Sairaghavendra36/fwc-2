\documentclass{article}
\usepackage{amsmath,amssymb,amsfonts,amsthm}
\usepackage{enumitem}
\usepackage{hyperref,xcolor}
\hypersetup{
    colorlinks,
    urlcolor={black}	%black!50!blue
}
\newcommand{\Mod}[1]{\ (\mathrm{mod}\ #1)}
\let\vec\mathbf

\def\inputGnumericTable{}
\usepackage{array}
\usepackage{longtable}
\usepackage{calc}
\usepackage{multirow}
\usepackage{hhline}
\usepackage{ifthen}

\providecommand{\cbrak}[1]{\ensuremath{\left\{#1\right\}}}
\newcommand{\Problem}{\noindent \textbf{Problem: }}
\newcommand{\solution}{\noindent \textbf{Solution: }}
\setlist[enumerate]{font=\small\bfseries}

\begin{document}


\title{Assignment: Probability}
\author{\Large T.Sai Raghavendra - FWC22087}
\date{}


\maketitle

\Problem A die has two faces each with number ‘1’, three faces each with number ‘2’ and
one face with number ‘3’. If die is rolled once, determine
\begin{enumerate}
\item[(i)] P(2)
\item[(ii)] P(1 or 3)
\item[(iii)] P(not 3)
\end{enumerate}

\solution
Total number of faces = 6\\
Let the faces of die be X = $\{1,2,3\}$.\\
Probability P = $\cfrac{Total number of favourable outcomes}{Total number of Possible outcomes}$.

	\begin{table}[h!]
	\small
	\centering
	%\subimport{../tables/}{table1.tex}
	\input{../tables/table1.tex}
	\caption{Probabilities of X}
	\label{tables:table1}
	\end{table}

\begin{enumerate}
\item[(i)] P(X=2) = $\cfrac{1}{2}$\\  %from \ref{tables:table1}
	
\item[(ii)] \begin{align}
P(X=1 + X=3)	=& \; P(1) + P(3) - P(1,3)\\
				=& \; \cfrac{1}{3} + \cfrac{1}{6}   (\therefore P(1,3) = 0)\\
				=& \; \cfrac{3}{6}\\
P(X=1 + X=3) 	=& \; \cfrac{1}{2}
\end{align}

\item[(iii)] \begin{align}
P(X=3)\prime =& \; 1 - P(3)\\
			 =& \; 1 - \cfrac{1}{6}\\
			 =& \; \cfrac{5}{6}
\end{align}
\end{enumerate}
\end{document}