\documentclass{article}
\usepackage{amsmath,amssymb,amsfonts,amsthm}
\providecommand{\pr}[1]{\ensuremath{\Pr\left(#1\right)}}
\usepackage{enumitem}
\usepackage{hyperref,xcolor}
\hypersetup{
    colorlinks,
    urlcolor={black}	%black!50!blue
}
\newcommand{\Mod}[1]{\ (\mathrm{mod}\ #1)}
\let\vec\mathbf

\def\inputGnumericTable{}
\usepackage{array}
\usepackage{longtable}
\usepackage{calc}
\usepackage{multirow}
\usepackage{hhline}
\usepackage{ifthen}

\providecommand{\brak}[1]{\ensuremath{\left\{#1\right\}}}
\newcommand*{\permcomb}[4][0mu]{{{}^{#3}\mkern#1#2_{#4}}}
%\newcommand*{\perm}[1][-3mu]{\permcomb[#1]{P}}
\newcommand*{\comb}[1][-1mu]{\permcomb[#1]{C}}
\newcommand{\Problem}{\noindent \textbf{Problem: }}
\newcommand{\solution}{\noindent \textbf{Solution: }}
\setlist[enumerate]{font=\small\bfseries}
\renewcommand\thefootnote{\textcolor{black}{\arabic{footnote}}}

\begin{document}


\title{Assignment: Probability}
\author{\Large T.Sai Raghavendra - FWC22087}
\date{}


\maketitle
\begin{enumerate}[label=13.\arabic{enumi}.\arabic{enumii}]%,ref=\thesection.\theenumi.\theenumi]
\numberwithin{equation}{enumi}
\numberwithin{table}{enumi}
%%%13.4.3
\setcounter{enumi}{3}
\setcounter{enumii}{3}

\item \footnote{Read question numbers as (CHAPTER NUMBER).(EXERCISE NUMBER).(QUESTION NUMBER)}Let X represent the difference between the number of heads and the number of tails obtained when a coin is tossed 6 times. What are possible values of X?\\

\solution

	\begin{table}[h]
	\centering
	\input{tables/table1.tex}
	\caption{Variable description.}
	\label{tables:table1}
	\end{table}
	
\begin{enumerate}
\item Number of heads in 6 tosses of a coin.
\begin{align}
p_{X_1}(k) &= 
\begin{cases}
\comb{n}{k}{p_1}^{k}{q_1}^{n-k} & 0 \le k \le 6\\               
\end{cases}
\end{align}
	
\item Number of tails in 6 tosses of a coin.
\begin{align}
p_{X_2}(k) &= 
\begin{cases}
\comb{n}{k}{p_2}^{k}{q_2}^{n-k} & 0 \le k \le 6\\              
\end{cases}
\end{align}
\end{enumerate}
Thus, the desired outcome is\\
\begin{align}
X = X_1 - X_2
\end{align}
\begin{align}
X(6H,0T) = |6-0| = 6\\
X(5H,1T) = |5-1| = 4\\
X(4H,2T) = |4-2| = 2\\
X(3H,3H) = |3-3| = 0\\
X(2H,4T) = |2-4| = 2\\
X(1H,5T) = |1-5| = 4\\
X(0H,6T) = |0-6| = 6
\end{align}
Thus, the possible values of X are 0,2,4 and 6.
\end{enumerate}
\end{document}
