\documentclass[journal,12pt,twocolumn]{IEEEtran}
\usepackage{graphicx}
\graphicspath{{./figs/}}{}
\usepackage{amsmath,amssymb,amsfonts,amsthm}
\newcommand{\myvec}[1]{\ensuremath{\begin{pmatrix}#1\end{pmatrix}}}
\usepackage{listings}
\usepackage{watermark}
\usepackage{titlesec}
\usepackage{amsmath}
\newcommand{\Mod}[1]{\ (\mathrm{mod}\ #1)}
\let\vec\mathbf

\titlespacing{\subsection}{0pt}{\parskip}{-3pt}
\titlespacing{\subsubsection}{0pt}{\parskip}{-\parskip}
\titlespacing{\paragraph}{0pt}{\parskip}{\parskip}
\newcommand{\figuremacro}[5]{
    
}
\lstset{
frame=single, 
breaklines=true,
columns=fullflexible
}
\thiswatermark{\centering \put(0,-105.0){\includegraphics[scale=0.08]{logo.jpg}} }

\sloppy
\title{\mytitle}
\title{
Probability Assignment
}
\author{T.Sai Raghavendra(FWC22087)}
\begin{document}
\maketitle
\tableofcontents
\bigskip
\section{\textbf{Problem2}}	
\subsection{\textbf{Question}}
If A and B are events such that P($\frac{A}{B}$) = P($\frac{B}{A}$), then\\
(A) A $\subset$ B but A $\not=$ B \\(B) A = B\\
(C) A $\cap$ B = $\phi$ \\(D) P(A) = P(B)\\
\subsection{\textbf{Solution}}
Given, P($\frac{A}{B}$) = P($\frac{B}{A}$)
\begin{center}
$\implies$ $\frac{P(AB)}{P(B)}$ = $\frac{P(BA)}{P(A)}$\\ 
$\implies$ $\frac{P(AB)}{P(B)}$ = $\frac{P(AB)}{P(A)}$ (Since P(AB) = P(BA))\\
$\implies$ $\frac{1}{P(B)}$ = $\frac{1}{P(A)}$\\
\end{center}
\begin{center}
$\therefore$ P(A) = P(B)
\end{center}
\end{document}